% Agradecimentos

\chapter*{Agradecimentos}

Eu gostaria, primeiramente, de externar os meus sinceros agradecimentos aos meus pais, que, durante toda a minha vida acadêmica e profissional, sempre me apoiaram, me ouviram e me orientaram nas mais difíceis decisões da vida.

Eu gostaria de agradecer, profundamente, à minha esposa Márcia Marillac Cardoso Oliveira e ao meu filho Mateus Cardoso Moreira pela paciência, amor e por compreenderem a dura jornada que enfrentei. Muito obrigado por me apoiarem em todos os momentos.

Gostaria, também, de expressar os meus sinceros agradecimentos ao meu orientador Uirá Kulezsa, pela oportunidade de trabalhar juntos e por me guiar sabiamente durante o período do mestrado. Em particular, para mim, o aprendizado foi ainda maior, pois pude amadurecer no tocante à relação aluno-professor. Eu, enquanto aluno e professor, poderei me espelhar e inspirar em diversas atitudes do professor e do orientador Uirá quando me relacionar com os meus alunos. Esse aprendizado será levado para toda a vida acadêmica e profissional.

Também gostaria de agradecer ao meu co-orientador Felipe Alves Pereira Pinto, por me dar a oportunidade de colaborar e ajuda-lo no seu trabalho de doutorado, quando eu ainda aspirava ser um aluno de mestrado. Além disso, durante a minha jornada, suas orientações, dicas e apoio foram fundamentais para a condução dos estudos e conclusão deste trabalho. Gostaria de agradecer, também, ao professor Christoph Treude, da escola de Ciências da Computação da Universidade de Adelaide, Austrália, que foi fundamental por nos guiar com relação ao modo de avaliação empregado neste trabalho. Seus conselhos e dicas nos guiaram na condução dos estudos e posterior análise dos resultados.

Por fim, gostaria de agradecer aos colegas do CASE Lab \& Research Group, em particular ao João Helis, e aos colegas do Laboratório LETS, em especial a Fábio Penha. Juntos, durante a nossa chamada ``hora do café'', pudemos discutir e compartilhar as nossas pesquisas, aliviar a pressão pelos prazos e, claro, nos divertir um pouco. Certamente, boas decisões e sugestões para este trabalho surgiram nesses momentos.