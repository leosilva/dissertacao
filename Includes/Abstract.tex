% Resumo em língua estrangeira (em inglês Abstract, em espanhol Resumen, em francês Résumé
\begin{center}
	{\Large{\textbf{\mscThesisEnglishTitle}}}
\end{center}

\vspace{1cm}

\begin{flushright}
	Author: \author\\
	Advisor: \advisor \\
	Co-advisor: \coadvisor
\end{flushright}

\vspace{1cm}

\begin{center}
	\Large{\textsc{\textbf{Abstract}}}
\end{center}

\noindent During the maintenance and evolution process of a software system, it can undergo several modifications, which can have negative consequences, reducing its quality and increasing its complexity. This deterioration can also affect system performance over time. Thus, without due monitoring, the performance quality attribute may no longer be adequately met. The software visualization area proposes the use of techniques whose objective is to improve the understanding of the software and to make its development process more productive. In this context, this work presents a tool to visualize performance deviations from subsequent evolutions of a software system to assist the analysis of performance evolution between software versions. The tool allows, through call graph and scenario summarization visualizations, that developers and architects can identify scenarios and methods that have had variations in their performance, including the potential causes of such deviations through commits. This work also presents an empirical study that evaluates the use of the tool by applying it to 10 evolutionary versions of 2 open-source systems from different domains and by submitting online questionnaires to obtain feedback from its developers and architects. The results of the conducted study bring preliminary evidence of the effectiveness of visualizations provided by the tool compared to tabular data. In addition, the nodes suppression algorithm of the call graph visualization was able to reduce between 73.77\% and 99.83\% the number of nodes to be displayed to the user, allowing him to be able to identify more easily the possible causes of variations.

\noindent\textit{Keywords}: software visualization, software evolution, performance.