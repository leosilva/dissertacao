% Resumo em língua estrangeira (em inglês Abstract, em espanhol Resumen, em francês Résumé
\begin{center}
	{\Large{\textbf{\mscThesisEnglishTitle}}}
\end{center}

\vspace{1cm}

\begin{flushright}
	Author: \author\\
	Advisor: \advisor \\
	Co-advisor: \coadvisor
\end{flushright}

\vspace{1cm}

\begin{center}
	\Large{\textsc{\textbf{Abstract}}}
\end{center}

\noindent During the maintenance process, continuous modifications can have negative consequences, reducing the quality of the software and increasing its complexity. This deterioration can also affect system performance over time. Thus, without due follow-up, the performance-quality attribute may no longer be matched with what was defined from architectural or design decisions. The software visualization area uses techniques that aim to improve software understanding and make your development process more productive. In this sense, this work presents a set of software visualizations with the purpose of helping to analyze the evolution of performance between software versions, so that developers and architects can identify scenarios and methods that had variations in their performance, including potential Causes of such deviations through commits. An empirical study was conducted applying the tool in two open-source systems from different domains and submitting questionnaires online to obtain feedback from its developers and architects. Among the results found, the information presented in the call graph visualization proved to be more effective for identifying the methods and causes of performance deviations than those presented in tabular format. Also, the nodes suppression algorithm of this visualization was able to reduce the number of nodes to be displayed to the user to 75\% of the analyzed scenarios, reducing the complexity of this visualization by 73.77\% and 99.83\%. Regarding the visualization of scenario summarization, it showed without prejudice the scenarios of 67\% of the analyzed versions.

\noindent\textit{Keywords}: software visualization, software evolution, performance.