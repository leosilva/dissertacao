% Resumo em língua estrangeira (em inglês Abstract, em espanhol Resumen, em francês Résumé
\begin{center}
	{\Large{\textbf{\mscThesisEnglishTitle}}}
\end{center}

\vspace{1cm}

\begin{flushright}
	Author: \author\\
	Advisor: \advisor \\
	Co-advisor: \coadvisor
\end{flushright}

\vspace{1cm}

\begin{center}
	\Large{\textsc{\textbf{Abstract}}}
\end{center}

\noindent The maintenance and evolution of software systems can bring several code changes that can potentially reduce their quality and increase their complexity. One critical quality attribute that be affected over time is the system performance. Thus, without due monitoring, the performance quality attribute may no longer be adequately addressed. The software visualization area proposes the use of techniques whose objective is to improve the understanding of the software and to make its development process more productive. In this context, this work presents PerfMiner Visualizer - a tool to visualize and analyze the performance deviations from subsequent evolutions of a software system. Through call graph and scenario summarization visualizations, the tool allows developers and architects identifying scenarios and methods that have variations in their performance, including the potential causes of such deviations through commits. This work also presents an empirical study that evaluates the use of the tool by applying it to 10 evolutionary versions of 2 open source systems from different domains and by submitting online questionnaires to obtain feedback from their developers and architects. The results of the conducted study bring preliminary evidence of the effectiveness of visualizations provided by the tool compared to tabular data. In addition, the nodes suppression algorithm of the call graph visualization was able to reduce between 73.77\% and 99.83\% the number of nodes to be displayed to the user, allowing him to be able to identify more easily the possible causes of variations.

\noindent\textit{Keywords}: software visualization, software evolution, performance.