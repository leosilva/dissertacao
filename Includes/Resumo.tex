% Resumo em língua vernácula
\begin{center}
	{\Large{\textbf{\mscThesisTitle}}}
\end{center}

\vspace{1cm}

\begin{flushright}
	Autor: \author\\
	Orientador(a): \advisor \\
	Coorientador(a): \coadvisor
\end{flushright}

\vspace{1cm}

\begin{center}
	\Large{\textsc{\textbf{Resumo}}}
\end{center}

\noindent Cada vez mais inseridas na sociedade mediante vários dispositivos, muitas das aplicações são complexas devido a sua larga escala, tornando a sua arquitetura também complexa para entender e manter. A área de visualização de software lança mão de técnicas cujo objetivo é melhorar o entendimento do software e tornar mais produtivo o seu processo de desenvolvimento. Durante o processo de manutenção, as mudanças podem ter consequências negativas, como afetar o desempenho dos sistemas ao longo do tempo. Assim, sem o devido acompanhamento, o desempenho pode deixar de ser correspondido com o que foi definido a partir de decisões arquiteturais ou de \textit{design}. Nesse sentido, ferramentas de \textit{profiling} e APM podem ser utilizadas para medir o desempenho, entretanto, falham ao prover métricas e visualizações adequadas para o acompanhamento da sua evolução. Este trabalho apresenta um conjunto de visualizações de software com o intuito de auxiliar a análise da evolução do desempenho entre versões de um software, de modo que desenvolvedores e arquitetos possam identificar métodos dos cenários que tiveram variações no seu desempenho. A avaliação se deu aplicando a ferramenta em dois sistemas \textit{open-source} de domínios diferentes e submetendo questionários online para obter feedback dos seus contribuidores. Dentre os resultados, podem ser citados que a visualização do grafo de chamadas se mostrou mais eficaz para identificação dos métodos e das causas dos desvios de desempenho do que os dados tabulares, e que o algoritmo de supressão de nós conseguiu reduzir, para 75\% dos cenários analisados, entre 73,77\% e 99,83\% a quantidade de nós a serem exibidos para o usuário, diminuindo a complexidade dessa visualização.

\noindent\textit{Palavras-chave}: arquitetura de software, visualização de software, evolução de software.