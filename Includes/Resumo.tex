% Resumo em língua vernácula
\begin{center}
	{\Large{\textbf{\mscThesisTitle}}}
\end{center}

\vspace{1cm}

\begin{flushright}
	Autor: \author\\
	Orientador(a): \advisor \\
	Coorientador(a): \coadvisor
\end{flushright}

\vspace{1cm}

\begin{center}
	\Large{\textsc{\textbf{Resumo}}}
\end{center}

\noindent A manutenção e a evolução dos sistemas de software podem trazer várias mudanças de código que podem potencialmente reduzir sua qualidade e aumentar sua complexidade. Um atributo de qualidade crítico que é afetado ao longo do tempo é o desempenho do sistema. Assim, sem o devido acompanhamento, esse atributo de qualidade pode deixar de ser atendido adequadamente. A área de visualização de software propõe o uso de técnicas cujo objetivo é melhorar o entendimento do software e tornar mais produtivo o seu processo de desenvolvimento. Neste contexto, este trabalho apresenta o PerfMiner Visualizer - uma ferramenta para visualizar e analisar desvios de desempenho em evoluções subsequentes de um sistema de software. Através de visualizações de grafos de chamadas e sumarização de cenários, a ferramenta permite que desenvolvedores e arquitetos possam identificar cenários e métodos que tiveram variações no seu desempenho, inclusive as potenciais causas desses desvios através dos \textit{commits}. O trabalho também apresenta um estudo empírico que avalia o uso da ferramenta aplicando-a em 10 versões de evolução de 2 sistemas \textit{open source} de domínios diferentes e submetendo questionários online para obter feedback dos seus desenvolvedores e arquitetos. Os resultados do estudo conduzido trazem evidências preliminares da eficácia das visualizações providas pela ferramenta em comparação com dados tabulares. Além disso, o algoritmo de supressão de nós da visualização do grafo de chamadas foi capaz de reduzir entre 73,77\% e 99,83\% a quantidade de nós a serem exibidos para o usuário, permitindo que ele possa identificar mais facilmente as possíveis causas das variações.

\noindent\textit{Palavras-chave}: visualização de software, evolução de software, desempenho.