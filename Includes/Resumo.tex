% Resumo em língua vernácula
\begin{center}
	{\Large{\textbf{\mscThesisTitle}}}
\end{center}

\vspace{1cm}

\begin{flushright}
	Autor: \author\\
	Orientador(a): \advisor \\
	Coorientador(a): \coadvisor
\end{flushright}

\vspace{1cm}

\begin{center}
	\Large{\textsc{\textbf{Resumo}}}
\end{center}

\noindent Durante o processo de manutenção e evolução de um sistema de software, o mesmo pode sofrer diversas modificações, as quais podem trazer consequências negativas, diminuindo a sua qualidade e aumentando sua complexidade. Essa deterioração também pode afetar o desempenho dos sistemas ao longo do tempo. Assim, sem o devido acompanhamento, o atributo de qualidade de desempenho pode deixar de ser atendido adequadamente. A área de visualização de software propõe o uso de técnicas cujo objetivo é melhorar o entendimento do software e tornar mais produtivo o seu processo de desenvolvimento. Neste contexto, este trabalho apresenta uma ferramenta de visualização de desvios de desempenho de evoluções subsequentes de um sistema de software com o intuito de auxiliar a análise da evolução do desempenho entre versões de um software. A ferramenta permite, através de visualizações de grafos de chamadas e sumarização de cenários, que desenvolvedores e arquitetos possam identificar cenários e métodos que tiveram variações no seu desempenho, inclusive as potenciais causas desses desvios através dos \textit{commits}. O trabalho também apresenta um estudo empírico que avalia o uso da ferramenta aplicando-a em 10 versões de evolução de 2 sistemas \textit{open-source} de domínios diferentes e submetendo questionários online para obter feedback dos seus desenvolvedores e arquitetos. Os resultados do estudo conduzido trazem evidências preliminares da eficácia das visualizações providas pela ferramenta em comparação com dados tabulares. Além disso, o algoritmo de supressão de nós da visualização do grafo de chamadas foi capaz de reduzir entre 73,77\% e 99,83\% a quantidade de nós a serem exibidos para o usuário, permitindo que ele possa identificar mais facilmente as possíveis causas das variações.

\noindent\textit{Palavras-chave}: visualização de software, evolução de software, desempenho.