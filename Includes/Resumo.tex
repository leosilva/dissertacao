% Resumo em língua vernácula
\begin{center}
	{\Large{\textbf{\mscThesisTitle}}}
\end{center}

\vspace{1cm}

\begin{flushright}
	Autor: \author\\
	Orientador(a): \advisor \\
	Coorientador(a): \coadvisor
\end{flushright}

\vspace{1cm}

\begin{center}
	\Large{\textsc{\textbf{Resumo}}}
\end{center}

\noindent As aplicações estão cada vez mais inseridas na sociedade mediante vários dispositivos. Muitas delas são complexas devido a sua larga escala, tornando a sua arquitetura também complexa para entender e manter. A área de visualização de software lança mão de técnicas cujo objetivo é melhorar o entendimento do software e tornar mais produtivo o seu processo de desenvolvimento. A inexistência de atividades para que os desenvolvedores e arquitetos possam entender a evolução arquitetural pode levar a sua degradação, fazendo com que os atributos de qualidade inicialmente definidos deixem de ser atendidos. Nesse sentido, em relação a medição do desempenho de aplicações, ferramentas de \textit{profiling} e APM podem ser utilizadas, entretanto, falham ao prover métricas e visualizações adequadas para o acompanhamento da evolução desse atributo de qualidade. Este trabalho apresenta um conjunto de visualizações de software com o intuito de auxiliar a análise da evolução do desempenho entre versões de determinado software, possibilitando que desenvolvedores e arquitetos identifiquem métodos dos cenários que degradaram ou melhoraram o seu desempenho. O conjunto de visualizações proposto será avaliado a partir de estudos empíricos realizados em aplicações \textit{open source}.

\noindent\textit{Palavras-chave}: arquitetura de software, visualização de software, evolução de software.