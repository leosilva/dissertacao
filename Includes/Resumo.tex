% Resumo em língua vernácula
\begin{center}
	{\Large{\textbf{\mscThesisTitle}}}
\end{center}

\vspace{1cm}

\begin{flushright}
	Autor: \author\\
	Orientador(a): \advisor \\
	Coorientador(a): \coadvisor
\end{flushright}

\vspace{1cm}

\begin{center}
	\Large{\textsc{\textbf{Resumo}}}
\end{center}

\noindent Durante o processo de manutenção, as modificações progressivas podem ter consequências negativas, diminuindo a qualidade do software e aumentando sua complexidade. Essa deterioração também pode afetar o desempenho dos sistemas ao longo do tempo. Assim, sem o devido acompanhamento, o atributo de qualidade de desempenho pode deixar de ser correspondido com o que foi definido a partir de decisões arquiteturais ou de \textit{design}. A área de visualização de software lança mão de técnicas cujo objetivo é melhorar o entendimento do software e tornar mais produtivo o seu processo de desenvolvimento. Nesse sentido, este trabalho apresenta um conjunto de visualizações de software com o intuito de auxiliar a análise da evolução do desempenho entre versões de um software, de modo que desenvolvedores e arquitetos possam identificar cenários e métodos que tiveram variações no seu desempenho, inclusive as potenciais causas desses desvios através dos \textit{commits}. Foi realizado um estudo empírico aplicando a ferramenta em dois sistemas \textit{open-source} de domínios diferentes e submetendo questionários online para obter feedback dos seus desenvolvedores e arquitetos. Dentre os resultados encontrados, podem ser citados que as informações apresentadads na visualização do grafo de chamadas se mostrou mais eficaz para identificação dos métodos e das causas dos desvios de desempenho do que as apresentadas em formato tabular. Além disso, o algoritmo de supressão de nós dessa visualização conseguiu reduzir, para 75\% dos cenários analisados, entre 73,77\% e 99,83\% a quantidade de nós a serem exibidos para o usuário, diminuindo a complexidade dessa visualização. Sobre a visualização da sumarização de cenários, ela mostrou sem prejuízos os cenários de 67\% das versões analisadas.

\noindent\textit{Palavras-chave}: visualização de software, evolução de software, desempenho.