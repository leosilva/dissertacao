% Capíulo 5
\chapter{Trabalhos Relacionados} \label{ch:trabalhos-relacionados}

Este capítulo confronta este trabalho com outras pesquisas que analisam a evolução do desempenho de sistemas. Qualquer trabalho ou ferramenta que mensure a evolução do atributo de qualidade de desempenho e possua visualizações para exibi-la é considerado relacionado a este. Na seção \ref{sec:trabalhos-relacionados-ferramentas-profiling} são comentadas as ferramentas de \textit{profiling}. Na seção \ref{sec:trabalhos-relacionados-ferramentas-apm} são mostradas as ferramentas APM. Por fim, na seção \ref{sec:trabalhos-relacionados-adordagens-degradacao-desempenho} são discutidas as abordagens de medição da degradação de desempenho.

Muitas abordagens relacionadas a visualização de software têm sido propostas para comparar versões de software de um ponto de vista geral da arquitetura \cite{Steinbruckner2010b}\cite{Telea2008}\cite{Collberg2003}\cite{Eick1992}\cite{Holten2008}. Outras abordagens focam em visualizar as métricas do software em diferentes versões \cite{Langelier2008}\cite{Lanza2001}\cite{Pinzger2005}\cite{Wettel2008}. No entanto, essas abordagens diferem deste trabalho uma vez que o objetivo é comparar um aspecto dinâmico do software, o desempenho, ao invés de aspectos estáticos ou estruturais.

\section{Ferramentas de Profiling} \label{sec:trabalhos-relacionados-ferramentas-profiling}

Há ferramentas de \textit{profiling} que podem realizar a medição do atributo de qualidade de desempenho, no entanto, com diferentes características. O \textit{JProfiler} \cite{JProfiler} e o \textit{YourKit Java Profiler} \cite{Profiler2016} são ferramentas comerciais para realizar profiling de aplicações na linguagem Java. \citeauthor{SandovalAlcocer2013} comentam algumas limitações dessas duas ferramentas:
\begin{itemize}
	\item \textit{Variações de desempenho têm que ser manualmente rastreadas}: para cada execução, o \textit{profiler} tem que ser manualmente configurado para executar uma versão em particular. Depois, os dados do \textit{profiling} podem ser salvos no sistema de arquivos. Após realizar esse procedimento por duas vezes, ambas as execuções podem ser comparadas. Entretanto, cada execução requer muito trabalho manual;
	\item \textit{Faltam métricas relevantes}: ambas as ferramentas não consideram se o código-fonte foi alterado ou não. Como consequência, variações de desempenho em métodos não modificados podem distrair o programador de identificar alterações de código que realmente introduziram as variações;
	\item \textit{Representações visuais ineficientes}: O \textit{JProfiler} e o \textit{YourKit Java Profiler} usam uma tabela textual incrementada com alguns ícones para indicar variações. Dessa forma, entender qual variação de desempenho decorre de mudanças de software requer um esforço significativo do programador.
\end{itemize}

Os autores comentam que essas ferramentas, apesar de serem úteis para acompanhar o desempenho geral, são ineficientes para saber a diferença dos tempos dos métodos e, muitas vezes, insuficientes para compreender as razões para a variação de desempenho. Há ainda uma ferramenta que acompanha a JVM, chamada \textit{VisualVM}, que possui as mesmas limitações comentadas anteriormente. Além disso, o \textit{VisualVM} não oferece a comparação entre execuções, tornando difícil a visualização da evolução do desempenho entre as versões do software, uma vez que teria que ser feita manualmente para cada método desejado.

\section{Ferramentas APM} \label{sec:trabalhos-relacionados-ferramentas-apm}

O trabalho de \citeauthor{Ahmed2016} realizou um estudo para verificar se as ferramentas de gerenciamento de desempenho de aplicações - APM - são eficazes na identificação de regressões de desempenho. Os autores definem regressão de desempenho quando as atualizações em um software provocam uma degradação no seu desempenho. As ferramentas utilizadas no estudo foram \textit{New Relic} \cite{Relic2016}, \textit{AppDynamics} \cite{Appdynamics}, \textit{Dynatrace} \cite{Dynatrace2016} e \textit{Pinpoint} \cite{Pinpoint2016}. Como resultado, eles mostram que a maioria das regressões inseridas no código-fonte foram detectadas pelas ferramentas. Contudo, o processo de identificação do método exato cujo código foi inserido foi mais complicado, sendo necessário bastante trabalho manual: os autores inspecionavam as transações (requisições) marcadas como lentas e, manualmente, comparavam os respectivos \textit{stacktraces} para verificar se a ferramenta indicava corretamente a regressão de desempenho.

A ferramenta e a extensão proposta neste trabalho é diferente das ferramentas apresentadas no trabalho de \citeauthor{Ahmed2016} por realizar a análise de duas versões do software alvo do estudo, por automatizar o processo de identificação da causa do desvio de desempenho, por prover visualizações adequadas à identificação dos desvios de desempenho, bem como exibindo dados adicionais dos nós do grafo, por mostrar a evolução global e por cenário do desempenho e por proporcionar aos desenvolvedores a identificação diretamente do código-fonte das causas do desvio de desempenho.

\section{Abordagens de Degradação de Desempenho} \label{sec:trabalhos-relacionados-adordagens-degradacao-desempenho}

O trabalho de \citeauthor{SandovalAlcocer2013} propõe o \textit{Performance Evolution Blueprint}, uma abordagem visual para entender a causa de degradações de desempenho, comparando o desempenho de duas versões do sistema. Trata-se de uma visualização onde formas e cores dos elementos visuais indicam valores de métricas e propriedades do software analisado. O trabalho utiliza a ferramenta \textit{Rizel} para medição de propriedades dos métodos, tais quais: suas métricas, quais métodos foram adicionados, removidos ou modificados, e o seu tempo e número de execução. A abordagem foi desenvolvida na linguagem de programação Pharo. A Figura \ref{fig:trabalhos-relacionados-performance-evolution-blueprint} apresenta essa abordagem. A ferramenta foi avaliada ao ser aplicada em um software chamado Roassal\footnote{\href{http://agilevisualization.com}{http://agilevisualization.com}}.

\begin{figure}[!htb]
   \centering
   \frame{\includegraphics[scale=0.58]{Imagens/trabalho_relacionado_performance_evolution_blueprint.png}}
   \textsf{\caption[Exemplo do \textit{Performance Evolution Blueprint}]{Exemplo do \textit{Performance Evolution Blueprint} \cite{SandovalAlcocer2013}.\label{fig:trabalhos-relacionados-performance-evolution-blueprint}}}
\end{figure}

A ferramenta {\textit{\toolName}} e o \textit{Performance Evolution Blueprint} utilizam a metáfora de grafo. No entanto, a primeira oferece as seguintes vantagens com relação à segunda:
\begin{itemize}
   \item \textit{Informações da Hierarquia de Chamadas}: no grafo de chamadas do {\textit{\toolName}} são exibidas informações como o tempo de execução total, próprio e de desvio para cada nó, o tempo de execução do cenário (com a porcentagem de variação) e os nomes de cada nó da hierarquia de chamadas (próximos ao nó desviado/adicionado/removido). Essas informações apresentadas pela ferramenta se mostram necessárias, como mencionado pelo participante P5GA: \textit{``o gráfico é bastante simples e fácil de entender. Não há muita informação, apenas o necessário.''}. Soma-se a isso, as informações contidas na seção de Sumário e Histórico da visualização. Com essas informações, torna-se factível identificar não só as possíveis causas dos desvios de desempenho, mas também a sua gravidade e localização na hierarquia de chamadas;
   \item \textit{Diferentes Elementos Visuais}: a visualização do grafo de chamadas fornece elementos visuais diferentes para exibir as informações. O nó de agrupamento mostra que existem mais nós que foram omitidos e não estão diretamente ligados aos indicados com desvios/adicionados/removidos. Na visualização proposta por \citeauthor{SandovalAlcocer2013}, esses nós são, por padrão, omitidos, mas podem ser exibidos através da interação do usuário. Outro elemento visual são as setas indicativas da porcentagem de degradação ou otimização dos nós, propiciando a identificação da gravidade do desvio de cada nó. Esse elemento visual foi mencionado no estudo, pelo participante P5GA: \textit{``Eu gostei .. da seta verde/vermelha indicando o nível de melhoria/degradação.''}. No \textit{Performance Evolution Blueprint}, o usuário pode ter acesso às porcentagens de desvio de um nó apenas ao passar o ponteiro do mouse sobre ele. Dessa forma, não há uma visão geral da gravidade do desvio de todos os nós indicados com variações de desempenho;
   \item \textit{Diferentes Funcionalidades}: o grafo de chamadas do {\textit{\toolName}} possui funcionalidades de zoom e destaque da hierarquia de chamada dos nós com desvio/adicionados/removidos. O participante P5GA disse, sobre a funcionalidade de destaque: ``Eu gostei do recurso de destaque.'';
   \item \textit{Maior Potencial de Aplicabilidade}: por ser compatível com a linguagem de programação Java, o potencial de aplicabilidade da ferramenta proposta por esta dissertação é maior. Para efeitos comparativos, no momento da escrita deste trabalho, existem no GitHub mais de 3 milhões e 300 mil projetos \textit{open source} implementados na linguagem Java, ao passo que na linguagem SmallTalk existem pouco mais de 2 mil;
   \item \textit{Maior Potencial de Utilização}: juntamente com a compatibilidade com a linguagem de programação Java, há maior potencial de utilização da ferramenta por parte dos desenvolvedores e arquitetos dos sistemas. A ferramenta pode ser oferecida como um serviço, pode ser integrada a ferramentas de integração contínua através de \textit{plugins} ou ser oferecida de maneira \textit{standalone};
   \item \textit{Avaliação com Participantes}: a avaliação com participantes realizada nesta dissertação é um diferencial com relação ao trabalho de \citeauthor{SandovalAlcocer2013}. A partir dessa avaliação foi possível obter valiosas respostas dos usuários com relação a utilidade das visualizações para identificação das possíveis causas de variações de desempenho e a sua aplicabilidade nos processos de desenvolvimento de software das equipes.
\end{itemize}

%O trabalho proposto se diferencia do de \citeauthor{SandovalAlcocer2013} pelo fato de: (i) ser compatível com a linguagem Java; (ii) mostrar mais informações sobre a hierarquia de chamadas; (iii) exibir a evolução de cada cenário entre as versões do softwares; e (iv) utilizar diferentes elementos visuais para destacar a evolução do desempenho na visualização do \textit{call graph} dos cenários.

\citeauthor{Bergel} propuseram uma abordagem cujo objetivo é comparar duas versões de um software para a identificação de gargalos de execução. As visualizações propostas foram \textit{Structural Distribution Blueprint} e \textit{Behavioral Distribution Blueprint}. Ambas exibem informações de execução como um \textit{call graph}, onde os nós são os métodos e as arestas são as invocações. Cada nó é renderizado como uma caixa e uma invocação é uma linha que une dois nós. A primeira visualização exibe métrica que indicam a distribuição do tempo de execução ao longo da estrutura estática de um programa. Já a segunda mostra as informações de tempo de execução junto com as chamadas de métodos. Essa abordagem considera apenas se um método gastou ou não mais tempo de execução do que na versão anterior. A Figura \ref{fig:trabalhos-relacionados-bergel-robbes-binder} exemplifica a visualização proposta pelos autores. A avaliação foi realizada ao aplicar a ferramenta no \textit{framework} Mondrian\footnote{\href{http://scg.unibe.ch/archive/papers/Meye06aMondrian.pdf}{http://scg.unibe.ch/archive/papers/Meye06aMondrian.pdf}}.

As principais vantagens do trabalho proposto por esta dissertação com relação ao de \citeauthor{Bergel} são:
\begin{itemize}
   \item \textit{Mais Métricas}: por adicionar mais métricas no grafo de chamadas, como a porcentagem de desvio de desempenho e se houve métodos adicionados ou removidos, o {\textit{\toolName}} fornece a possibilidade de o usuário identificar as possíveis causas e a gravidade dos desvios de desempenho. Essas métricas adicionadas fortalecem a análise da evolução do sistema ao longo das versões;
   \item \textit{Agrupamento de Nós}: no {\textit{\toolName}}, o nó de agrupamento mostra que existem mais nós na hierarquia de chamadas que foram omitidos e não estão diretamente ligados aos indicados com desvios/adicionados/removidos. A proposta de \citeauthor{Bergel} não menciona sobre escalabilidade em nenhuma das suas visualizações, nem como essa situação é trata;
   \item \textit{Identificação das Causas dos Desvios}: o grafo de chamadas do {\textit{\toolName}} indica os \textit{hashes} dos \textit{commits} que possivelmente foram os responsáveis pelos desvios de desempenho dos nós e, consequentemente, do cenário. As visualizações \textit{Structural Distribution Blueprint} e \textit{Behavioral Distribution Blueprint} não fornecem essa possibilidade;
   \item \textit{Verificação das Mudanças}: outra vantagem é que, uma vez que os \textit{hashes} dos \textit{commits} são exibidos para os nós com desvios/adicionados/removidos, eles podem ser clicados e, assim, o usuário tem acesso às mudanças implementadas no código-fonte que podem ter causado tal desvio de desempenho;
   \item \textit{Visão Geral dos Cenários}: \citeauthor{Bergel} não mencionam se há uma forma de se obter uma visão geral dos gargalos de desempenho, para duas versões de um software, encontrados pela abordagem. Para essa finalidade, o {\textit{\toolName}} fornece a visualização de sumarização de cenários.
\end{itemize}

\begin{figure}[!htb]
   \centering
   \frame{\includegraphics[scale=0.60]{Imagens/trabalho_relacionado_bergel_robbes_binder.png}}
   \textsf{\caption[Exemplo da visualização comportamental proposta por Bergel, Robbes e Binder.]{Exemplo da visualização comportamental proposta por \citeauthor{Bergel}.\label{fig:trabalhos-relacionados-bergel-robbes-binder}}}
\end{figure}

\citeauthor{Mostafa2009} propõem uma técnica que compara duas árvores de contexto de chamadas \abrv[CCT -- \textit{Context Call Tree}]{}(CCT, do inglês \textit{Context Call Tree}), cada uma obtida de uma versão diferente de um software. Os autores apresentam o \textit{PARCS}, uma ferramenta de análise que identifica automaticamente diferenças entre o comportamento da execução de duas revisões de uma aplicação. A abordagem usa como base o algoritmo de correspondência de árvores comuns para comparar duas CCTs. No entanto, o suporte visual usado pelo PARCS não representa adequadamente a variação de uma estrutura dinâmica e múltiplas métricas \cite{Pablo2013}. Outra limitação dessa abordagem é que ela não detecta nós adicionados. Quando comparadas, duas CCTs que diferem apenas em um nó adicionado são consideradas completamente diferentes pelo PARCS. A Figura \ref{fig:trabalhos-relacionados-parcs} exibe um exemplo dessa abordagem. A avaliação do PARCS foi realizada aplicando-o na ferramenta chamada FindBugs\footnote{\href{http://findbugs.sourceforge.net}{http://findbugs.sourceforge.net}}.

\begin{figure}[!htb]
   \centering
   \frame{\includegraphics[scale=0.60]{Imagens/trabalho_relacionado_parcs.png}}
   \textsf{\caption[Exemplo do PARCS.]{Exemplo do PARCS \cite{Mostafa2009}.\label{fig:trabalhos-relacionados-parcs}}}
\end{figure}

O trabalho proposto nesta dissertação oferece as seguintes vantagens ao PARCS, dentre outras:
\begin{itemize}
   \item \textit{Nós Adicionados e Removidos}: o grafo de chamadas do {\textit{\toolName}} destaca os nós adicionados e removidos na hierarquia de chamadas de cada cenário. Esses nós são fundamentais para o entendimento da real causa dos desvios de desempenho de determinado cenário ao longo das versões;
   \item \textit{Mais Métricas}: de maneira semelhante ao trabalho relacionado comentado anteriormente, por adicionar mais métricas no grafo de chamadas, como a porcentagem de desvio de desempenho, os tempos de execução total, próprio e de desvio para cada nó, o {\textit{\toolName}} fornece a possibilidade de o usuário identificar as possíveis causas e a gravidade dos desvios de desempenho;
   \item \textit{Identificação das Causas dos Desvios}: diferentemente do trabalho de \citeauthor{Mostafa2009}, o grafo de chamadas do {\textit{\toolName}} indica os \textit{hashes} dos \textit{commits} que possivelmente foram os responsáveis pelos desvios de desempenho dos nós e, consequentemente, do cenário;
   \item \textit{Verificação das Mudanças}: por indicar os \textit{hashes} dos \textit{commits}, ao clicar nesses \textit{hashes}, o usuário pode ter acesso às mudanças implementadas no código-fonte que podem ter causado o desvio de desempenho do cenário;
   \item \textit{Visão Geral dos Cenários}: os autores não mencionam se há uma forma de se obter uma visão geral das comparações entre as revisões da aplicação. A visualização da sumarização de cenários fornecida pelo {\textit{\toolName}} possibilita ao usuário obter uma visão geral de todos os cenários para a análise de um par de versões de um software.
\end{itemize}

\citeauthor{Bezemer2015} realizam a comparação do desempenho de duas versões de um software através de gráficos de chama diferenciais (do inglês, \textit{differential flame graphs} -- DFG). Dadas duas versões v1 (anterior) e v2 (atual), a ferramenta realiza a comparação de três maneiras: (i) com v1 como base, (ii) com v2 como base e (iii) apenas as diferenças do item (ii). Utiliza cores para destacar as comparações: o branco indica que não houve mudanças, o azul que houve melhora no desempenho e o vermelho indica que este piorou. Entretanto, a abordagem não indica as causas das variações de desempenho e os autores indicam que o principal desafio é a coleta dos dados, pois são necessárias ferramentas de \textit{profiling} de terceiros para a análise das versões e eles destacam que pode ser difícil coletar os dados para algumas linguagens, como o Java e Python. A Figura \ref{fig:trabalhos-relacionados-dfg} mostra o DFG. A avaliação foi feita aplicando a abordagem no software utilitário Rsync\footnote{\href{https://rsync.samba.org}{https://rsync.samba.org}}.

\begin{figure}[!htb]
   \centering
   \frame{\includegraphics[scale=0.42]{Imagens/trabalho_relacionado_dfg.png}}
   \textsf{\caption[Exemplo do DFG.]{Exemplo do DFG \cite{Bezemer2015}.\label{fig:trabalhos-relacionados-dfg}}}
\end{figure}

O {\textit{\toolName}} apresenta vantagens com relação ao DFG, embora a metáforas visuais utilizadas sejam diferentes:
\begin{itemize}
   \item \textit{Mais Métricas}: o DFG apresentado pelos autores apresentam poucas métricas com relação aos métodos executados. O trabalho proposto por esta dissertação mostra os tempos de execução total, próprio e o desvio para cada nó, a porcentagem do desvio, apresenta mais claramente os nomes dos nós na hierarquia de chamadas, além das informações que constam nas seções de Sumário e Histórico da visualização;
   \item \textit{Nós Adicionados e Removidos}: o {\textit{\toolName}} destaca os nós adicionados e removidos na hierarquia de chamadas de cada cenário, diferentemente do DFG. Como comentado anteriormente, a identificação desses nós são fundamentais para o entendimento da real causa dos desvios de desempenho de determinado cenário ao longo das versões;
   \item \textit{Agrupamento de Nós}: no {\textit{\toolName}}, o nó de agrupamento mostra que existem mais nós na hierarquia de chamadas que foram omitidos e não estão diretamente ligados aos indicados com desvios/adicionados/removidos. O DFG mostra todo rastreamento da pilha de chamadas dos métodos, de modo que se essa pilha for extensa, torna-se muito difícil, ou até mesmo inviável, identificar e distinguir os seus elementos;
   \item \textit{Identificação das Causas dos Desvios}: diferentemente do trabalho de \citeauthor{Bezemer2015}, o {\textit{\toolName}} indica os \textit{hashes} dos \textit{commits} que possivelmente foram os responsáveis pelos desvios de desempenho dos nós e, consequentemente, do cenário;
   \item \textit{Verificação das Mudanças}: por indicar os \textit{hashes} dos \textit{commits}, ao clicar nesses \textit{hashes}, o usuário pode ter acesso às mudanças implementadas no código-fonte que podem ter causado o desvio de desempenho do cenário;
   \item \textit{Visão Geral dos Cenários}: Os autores não mencionam se, através da abordagem, é possível obter uma visão geral das comparações realizadas. A visualização da sumarização de cenários fornecida pelo {\textit{\toolName}} possibilita uma visão geral de todos os cenários para a análise de um par de versões de um software.
\end{itemize}